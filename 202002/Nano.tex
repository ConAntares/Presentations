\documentclass [xcolor=svgnames, t] {beamer} 
\usepackage[utf8]{inputenc}
\usepackage{booktabs, comment} 
\usepackage[absolute, overlay]{textpos} 
\useoutertheme{infolines} 
\setbeamercolor{title in head/foot}{bg=internationalorange}
\setbeamercolor{author in head/foot}{bg=dodgerblue}
\usepackage{csquotes}
\usepackage[style=verbose-ibid,backend=bibtex]{biblatex}
\bibliography{bibfile}

\usepackage{amsmath}
\usepackage[makeroom]{cancel}


\usepackage{textpos}

\usepackage{tikz}

\usetheme{Madrid}
\definecolor{myuniversity}{RGB}{0, 60, 113}
\definecolor{internationalorange}{RGB}{231, 93,  42}
 	\definecolor{dodgerblue}{RGB}{0, 119,202}
\usecolortheme[named=myuniversity]{structure}
\usepackage{tikz}



\title[Turbulence Modeling (Group Project)]{Turbulence Modeling using OpenFOAM}
\subtitle{(Introduction to Turbulence - ENGR5005G)}
\institute[]{Mechanical Engineering \\Ontario Tech University }
\titlegraphic{\includegraphics[height=2.5cm]{ontario.png}}
\author[Arup Jyoti Chutia]{
	Arup Jyoti Chutia ,
	Brayden York and
	Marcus Ebert }


\institute[]{Mechanical Engineering \\Ontario Tech University }
\date{\today}


\addtobeamertemplate{navigation symbols}{}{%
    \usebeamerfont{footline}%
    \usebeamercolor[fg]{footline}%
    \hspace{1em}%
    \insertframenumber/\inserttotalframenumber
}

\begin{document}
\begin{frame}
 \titlepage   
\end{frame}




%%%%%%%%%%%%%%%%%%%%%%%%%%%%
\logo{\includegraphics[scale=0.2]{ontario.png}~%
}


%%%%%%%%%%%%%%%%%%%%%%%%%%



\begin{frame}{Outline}
\vspace{1cm}
\begin{center}
   \begin{itemize}
     \item Introduction
     \item Objectives
     \item Governing equations
     \item OpenFOAM solvers
     \item Geometry and problem parameters
     \item Mesh and boundary conditions
     \item Results
     \item Conclusions and recommendations
     \item References
 \end{itemize} 
\end{center}
 
\end{frame}


\begin{frame}{Introduction }
Turbulent flow 
 \begin{center}
    

\begin{columns}[onlytextwidth]
\column{0.5\textwidth}
\begin{itemize}
\item Chaotic changes in field values :
\begin{itemize}
    \item velocity
    \item pressure
\end{itemize}
\item High Reynolds number flow :
\vspace{2.5mm}
\begin{itemize}
    \item low momentum diffusion ($\mu$)
    \item high momentum convection  
\end{itemize}

%\item Mathematical models !




\end{itemize}



\column{0.5\textwidth}
\begin{figure}
    \centering
    \includegraphics[width=0.9\textwidth]{turb.jpg}
    \caption{1. Flow visualisation (source: www.bronkhorst.com).}
  %  \label{fig:my_label}
\end{figure}

\end{columns}

\end{center}

  
\end{frame}

\begin{frame}{Introduction (contd...)}
\begin{center}
    

\begin{columns}[onlytextwidth]
\column{0.5\textwidth}
%\vspace{1 cm}
Why turbulence modeling?\\
\vspace{1cm}
\begin{itemize}

\item  No general analytical theory
\item  Chaotic flow 
\item  Closure Problem

\item Mathematical models.

\end{itemize}
\column{0.5\textwidth}
\begin{figure}
    \centering
    \includegraphics[width=0.9\textwidth]{vel.png}
    \caption{2.(a)Laminar and (b) turbulent velocity (source: https://nptel.ac.in).}
  %  \label{fig:my_label}
\end{figure}
\end{columns}

\end{center}



\end{frame}



\begin{frame}{Objectives}
 \vspace{1.5cm}   
\begin{itemize}
    \item Understanding turbulence models in CFD (OpenFOAM).
    \item Simulations for transient and steady state conditions.
    \item Selecting  turbulence model.
\end{itemize}
\end{frame}

\begin{frame}{Governing equations (Mean flow)\autocite{Lecture3}}
  
 \textit{\textbf{RANS} equations for incompressible flow}:\\
  \textbf{Continuity equation}
  
     \begin{equation}
      \frac{\partial \overline{u_{i}}}{\partial x_{i}}=0
  \end{equation} 
  
 
  
  
  \textbf{Momentum equations }
  \begin{equation}
      \frac{\partial \overline{u_{i}} }{\partial t} + \overline{u_{j}} \frac{\partial \overline{u_{i}} }{\partial x_{j}} = -\frac{1}{\rho}\frac{\partial \overline{P}}{\partial x_{i}}+\nu\frac{\partial ^2\overline{u_{i}}}{\partial x_{j}\partial x_{j}}-\frac{\partial \overline{u_{i}'u_{j}'}}{\partial x_{j}}+\overline{g_{i}}
  \end{equation}
  
   \textbf{Scaler equation }
  \begin{equation}
     \frac{\partial \overline{\phi}}{\partial t}+\overline{u_{i}}\frac{\partial \overline{\phi}}{\partial x_{i}}=\frac{\partial}{\partial x_{i}}(D\frac{\partial \overline{\phi}}{\partial x_{i}})-\frac{\partial (\overline{{u_{i}^{'}\phi^{'}}})}{\partial x_{i}}
  \end{equation}
  
  
\end{frame}


\begin{frame}{Standard \textit{k-$\epsilon$} model\autocite{versteeg2007introduction}}

$\nu_{eff}=\nu+\nu_{t},\,\, \nu_{t}=?$
\begin{itemize}
    

\item k-turbulent kinetic energy

\begin{equation}
   k= \frac{1}{2}(\overline{u'^2}+\overline{v'^2}+\overline{w'^2})
\end{equation}
\item $\epsilon$ -turbulent dissipation
  \begin{itemize}
    \item rate of dissipation of $k$.
\end{itemize} 
\item Turbulent viscosity
\begin{itemize}
    \item  $\nu_{t}=0.09\frac{k^2}{\epsilon}$ 
\end{itemize}
\item Transport equations for $k$ and $\epsilon$
 \end{itemize}   
\end{frame}





\begin{frame}{OpenFOAM solvers }
   OpenFOAM solvers used in this project
   \vspace{1cm}
   \begin{itemize}
       \item \textbf{simpleFoam\autocite{chapter6}}: for steady state simulation.
       \begin{itemize}
           \item RAS models: \textit{kEpsilon, kOmega} and \textit{LRR}.
       \end{itemize}
       \item \textbf{pisoFoam\autocite{chapter7}}:  transient simulation for incompressible flow.
       \begin{itemize}
          
           \item LES models : \textit{Smagorinsky,
          kEqn}.
          \item RAS model: \textit{kEpsilon}
     
       \end{itemize}
   \end{itemize}
   
\end{frame}

\begin{frame}{Geometry and problem parameters } 
   %\textbf{Geometry and problem parameters }
   \begin{figure}
       \centering
       \includegraphics[width=0.9\textwidth]{pitz.PNG}
       \caption{3.Schematic of geometry used for simulations (source: http://training.uhem.itu.edu.tr). }
       %\label{fig:my_label}
   \end{figure}
   
   
\end{frame}

\begin{frame}{ Mesh and boundary conditions}
   \begin{figure}
       \centering
       \includegraphics[width=0.9\textwidth]{mesh.png}
       \caption{4.Hexahedral mesh (source:www.cfdsupport.com).}
      % \label{fig:my_label}
   \end{figure} 
   
  Velocity  boundary conditions
   \begin{itemize}
       \item \textit{Inlet}: Dirichlet
condition.
       \item \textit{Outlet}: Zero-gradient condition.
      \item  \textit{Upper Wall}: No slip .
       \item \textit{Bottom Wall}: No slip.
      % \item \textit{Side Walls}: No slip.
   \end{itemize}
   
\end{frame}

\begin{frame}{Turbulence – Steady State: Results}

\begin{figure}
    \centering
    \includegraphics[width=0.65\textwidth]{steady_velocity.png}
    \caption{5.Velocity magnitude for \textit{kEpsilon, kOmega} \,\,and \text{LRR} models.}
   % \label{fig:my_label}
\end{figure}
    
\end{frame}

\begin{frame}{Turbulence – Steady State: Results(contd...)}

\begin{figure}
    \centering
   \includegraphics[width=0.65\textwidth]{steady_nut.png}
    \caption{6.Turbulent viscosity for \textit{kEpsilon, kOmega}  \,\, and \text{LRR} models.}
   %\label{fig:my_label}
\end{figure}
    
\end{frame}


\begin{frame}{Turbulence - Transient : Results (Smargorinsky model)}
 \begin{figure}
    \centering
    \includegraphics[width=0.62\textwidth]{smagorinsky_velocity.png}
    \caption{7.Smargorinsky velocity magnitude at different time steps.}
   % \label{fig:my_label}
\end{figure}
       
\end{frame}

\begin{frame}{Turbulence - Transient : Results (Smargorinsky model)}
 \begin{figure}
    \centering
    \includegraphics[width=1\textwidth]{FlowField.png}
    \caption{8.Streamlines at 0.2s for Smargorinsky model.}
   % \label{fig:my_label}
\end{figure}
       
\end{frame}



\begin{frame}{Turbulence - Transient : Results (kEpsilon)}
    
    \begin{figure}
    \centering
    \includegraphics[width=0.62\textwidth]{kEpsilon_velocity.png}
    \caption{9.\textit{kEpsilon} model - Velocity magnitude at different
time steps

.}
    \label{fig:my_label}
\end{figure}
    
\end{frame}

\begin{frame}{Turbulence - Transient : Results (contd...)}
    
    \begin{figure}
    \centering
    \includegraphics[width=0.65\textwidth]{transient_velocity_02.png}
    \caption{10.Velocity vectors for different turbulence models - at
0.2s}
  %  \label{fig:my_label}
\end{figure}
    
\end{frame}

\begin{frame}{Turbulence - Transient : Results (contd...)}
    
    \begin{figure}
    \centering
    \includegraphics[width=0.6\textwidth]{nutturb.png}
    \caption{11.Turbulent viscosity for different turbulence models - at
0.2s}
  %  \label{fig:my_label}
\end{figure}
    
\end{frame}

\begin{frame}{Conclusions and Recommendations}
\textbf{Steady State Simulations}
\begin{itemize}
    \item Similar results for  \textit{kEpsilon, kOmega}\,\,and \textit{LRR}.
    
\end{itemize}
\textbf{Transient Simulations}
\begin{itemize}
    \item LES -(\textit{Smagorinsky, kEqn)} detail, fluctuation based.
    \item RAS -(\textit{kEpsilon}), averaging nature.
\end{itemize}
    
\end{frame}
\begin{frame}{References}
   \printbibliography
\end{frame}

\end{document}